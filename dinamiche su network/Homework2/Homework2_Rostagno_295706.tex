\documentclass[a4paper,12pt]{article}
\usepackage{graphicx}
\usepackage{verbatim}
\usepackage{amsmath}
\usepackage{enumitem}
\begin{document}
	\title{\textbf{Homework 2}}
	\author{Rostagno 295706}
	\date{\today}
	\maketitle
	
	\centering \textbf{Esercizio 1}\\
	\begin{itemize}
		\item \textbf{Punto a: } 
		Considerando la forma della matrice $P$ e della matrice $P_{lazy}$ date dall'esercizio, possiamo analizzare le convergenze come:\\
		\begin{itemize}
			\item \textbf{Convergenza della dinamica} $x(t+1)=Px(t)$:\\
			La matrice \(P\) è stocastica, con autovalore dominante \(\lambda_1 = 1\). Gli altri autovalori soddisfano \(|\lambda_i| < 1\) per \(i > 1\), garantendo la convergenza della dinamica.
			Quando \(t \to \infty\), la distribuzione \(\mathbf{x}(t)\) converge all'equilibrio, ovvero tutti i nodi assumono lo stesso valore, determinato dalla media pesata dello stato iniziale in base alla distribuzione invariante \(\pi\):
			\[
			x_i(\infty) = \pi^\prime \mathbf{x}(0), \quad \forall i \in V.
			\]
			\item \textbf{Convergenza della dinamica} $x(t+1)=P_{lazy}x(t)$:\\
			La matrice \(P_{\text{lazy}} = \frac{1}{2}(I + P)\) è anch'essa stocastica e conserva le proprietà di convergenza. Tuttavia, gli autovalori diversi da \(\lambda_1 = 1\) vengono "diminuiti" nella forma:
			\[
			\lambda_i^{\text{lazy}} = \frac{1}{2}(1 + \lambda_i),
			\]
			dove \(\lambda_i\) sono gli autovalori di \(P\).\\
			Questo rallentamento implica che il tempo necessario per raggiungere l'equilibrio sia maggiore rispetto alla dinamica \(x(t + 1) = P x(t)\).	
		\end{itemize}
		\item \textbf{Punto b: }
		Dobbiamo calcolare $\lambda_2$ della matrice $P_{lazy}$ e determinarne il tempo di rilassamento in funzione di $n \rightarrow \infty$. Sappiamo che $R_n$ è un grafo circolare, quindi gli autovalori della matrice di adiacenza W possono essere calcolati come:\\
		\[
		\mu_k = \sum_{j=0}^{n-1} c_j \omega_k^j 
		\]
		Dove $c_j$ è la prima riga della matrice W mentre $\omega_k = exp{\frac{2 \pi i }{n}k}$. Nel nostro caso $c_j$ ha gli 1 della prima riga posizionati nelle posizioni: $j=1$, $j=\frac{n}{2}$ e $j=n-1$.\\
	    Sviluppando la sommatoria otteniamo\\
	    \[
	    \omega_k + \omega_k^{n-1} + \omega_k^{\frac{n}{2}}
	    \]
	    che diventa\\
	    \[
	    \exp\left(\frac{2 \pi i k}{n}\right) + \exp\left(\frac{2 \pi (n-1) i k}{n}\right) + \exp\left(\pi i k\right)
	    \]
	    Sviluppando le forme trigonometriche otteniamo\\
	    \[
	    2 \cos(\frac{2 \pi k}{n}) + exp(i \pi k)
	    \]
		Essendo il grafo 3-regolare, \( P = W / 3 \), dunque lo spettro della matrice \( P \) si può ottenere come:
		\[
		\sigma(P) = \left\{\lambda_k = \frac{2}{3} \cos\left(\frac{2 \pi k}{n}\right) + \frac{1}{3} \exp(i \pi k), \quad k = 0, \ldots, n-1 \right\}.
		\]
		
		Di conseguenza lo spettro della matrice $P_{lazy}$:
		\[
		\sigma(P_{lazy}) = \left\{\lambda_k = \frac{1}{2} + \frac{1}{2}(\frac{2}{3} \cos\left(\frac{2 \pi k}{n}\right) + \frac{1}{3} \exp(i \pi k)) , \quad k = 0, \ldots, n-1 \right\}.
		\]
		
		Il secondo autovalore dominante di \( Q \) è:
		\[
		\lambda_2 = \frac{1}{3} + \frac{1}{3} \cos\left(\frac{2 \pi}{n}\right),
		\]
		corrispondente a \( k = 1 \). Ricordando lo sviluppo in serie di McLaurin (\( t \to 0 \)):
		\[
		\cos(t) = 1 - \frac{1}{2}t^2 + o(t^3),
		\]
		studiamo il comportamento asintotico di \( \lambda_2 \), e di conseguenza di \( \tau_{\text{rel}} = \frac{1}{1 - \lambda_2} \), per \( n \to \infty \):
		\[
		\lambda_2 = \frac{1}{3} + \frac{1}{3}\cos\left(\frac{2 \pi}{n}\right) = \frac{1}{3} + \frac{1}{3}\left(1 - \frac{2 \pi^2}{n^2} + \frac{o(1)}{n^3}\right) \approx \frac{2}{3} - \frac{2 \pi^2}{3 n^2}.
		\]
		
		Quindi:
		\[
		\tau_{\text{rel}} = \frac{1}{1 - \lambda_2} \approx \frac{3 n^2}{n^2 + 2 \pi^2}.
		\]
		\item \textbf{Punto c: }
		\item \textbf{Punto d: }
	\end{itemize} 
	
	
\end{document}