\documentclass[a4paper,12pt]{article}
\usepackage{graphicx}
\usepackage{verbatim}
\usepackage{amsmath}
\usepackage{enumitem}
\usepackage{amsfonts}
\usepackage{amssymb}
\usepackage{listings}
\usepackage{xcolor}




\begin{document}
	\title{\textbf{Homework2 Rostagno}}
	\author{295706}
	\date{\today}
	\maketitle
	
	\begin{itemize}
		\item \textbf{Esercizio 1: }
		\begin{enumerate}[label=\alph*)]
			\item Per determinare se la catena è esplosiva dobbiamo verificare che 
			\[
			\sum_{n=0}^\infty \frac{1}{\lambda_n} < \infty
			\]
			dove $\lambda_n(x)=q(x,x+1) + q(x,x-1)$.\\
			Nel nostro caso abbiamo:\\
			\[
			\lambda(x) =
			\begin{cases} 
				q(x, x+1) = 2^x, & \text{se } x = 0, \\ 
				q(x, x+1) + q(x, x-1) = 2^x + 2^{x+1} = 3 \cdot 2^x, & \text{se } x \geq 1.
			\end{cases}
			\]
			La sommatoria da verificare è:
			\[
			\sum_{x=0}^\infty \frac{1}{\lambda(x)}.
			\]
			
			Per \(x = 0\):
			\[
			\frac{1}{\lambda(0)} = \frac{1}{2^0} = 1.
			\]
			
			Per \(x \geq 1\):
			\[
			\frac{1}{\lambda(x)} = \frac{1}{3 \cdot 2^x}.
			\]
			
			La sommatoria diventa:
			\[
			\sum_{x=0}^\infty \frac{1}{\lambda(x)} = 1 + \sum_{x=1}^\infty \frac{1}{3 \cdot 2^x}.
			\]
			
			Calcoliamo la serie \(\sum_{x=1}^\infty \frac{1}{3 \cdot 2^x}\):
			\[
			\sum_{x=1}^\infty \frac{1}{3 \cdot 2^x} = \frac{1}{3} \sum_{x=1}^\infty \frac{1}{2^x}.
			\]
			Notiamo che la serie è una geometrica e quindi ne conosciamo la convergenza\\
			\[
			\sum_{x=1}^\infty \frac{1}{2^x} = \frac{\frac{1}{2}}{1-\frac{1}{2}}=1
			\]
			Abbiamo in conclusione che la serie diventa\\
			\[
			\sum_{x=0}^\infty \frac{1}{\lambda(x)} = 1 + \sum_{x=1}^\infty \frac{1}{3 \cdot 2^x}=1+\frac{1}{3}=\frac{4}{3}
			\]
			Quindi la catena è esplosiva.
			\item Ripetiamo gli stessi passaggi del punto precedente\\
			\[
			\lambda(x) =
			\begin{cases} 
				q(x, x+1) = 2^x, & \text{se } x = 0, \\ 
				q(x, x+1) + q(x, x-1) = 2^x + 2^{x} = 2 \cdot 2^x, & \text{se } x \geq 1.
			\end{cases}
			\]
			La sommatoria da verificare è:
			\[
			\sum_{x=0}^\infty \frac{1}{\lambda(x)}.
			\]
			
			Per \(x = 0\):
			\[
			\frac{1}{\lambda(0)} = \frac{1}{2^0} = 1.
			\]
			
			Per \(x \geq 1\):
			\[
			\frac{1}{\lambda(x)} = \frac{1}{2 \cdot 2^x}.
			\]
			
			La sommatoria diventa:
			\[
			\sum_{x=0}^\infty \frac{1}{\lambda(x)} = 1 + \sum_{x=1}^\infty \frac{1}{2 \cdot 2^x}.
			\]
			
			Calcoliamo la serie \(\sum_{x=1}^\infty \frac{1}{2 \cdot 2^x}\):
			\[
			\sum_{x=1}^\infty \frac{1}{2 \cdot 2^x} = \frac{1}{2} \sum_{x=1}^\infty \frac{1}{2^x}.
			\]
			Notiamo che la serie è una geometrica e quindi ne conosciamo la convergenza\\
			\[
			\sum_{x=1}^\infty \frac{1}{2^x} = \frac{\frac{1}{2}}{1-\frac{1}{2}}=1
			\]
			Abbiamo in conclusione che la serie diventa\\
			\[
			\sum_{x=0}^\infty \frac{1}{\lambda(x)} = 1 + \sum_{x=1}^\infty \frac{1}{2 \cdot 2^x}=1+\frac{1}{2}=\frac{3}{2}
			\]
			Quindi la catena è esplosiva.
			\item Ripetiamo gli stessi passaggi del punto precedente\\
			\[
			\lambda(x) =
			\begin{cases} 
				q(x, x+1) = 2^{x+1}, & \text{se } x = 0, \\ 
				q(x, x+1) + q(x, x-1) = 2^x + 2^{x+1} = 3 \cdot 2^x, & \text{se } x \geq 1.
			\end{cases}
			\]
			La sommatoria da verificare è:
			\[
			\sum_{x=0}^\infty \frac{1}{\lambda(x)}.
			\]
			
			Per \(x = 0\):
			\[
			\frac{1}{\lambda(0)} = \frac{1}{2^{0+1}} = \frac{1}{2}.
			\]
			
			Per \(x \geq 1\):
			\[
			\frac{1}{\lambda(x)} = \frac{1}{3 \cdot 2^x}.
			\]
			
			La sommatoria diventa:
			\[
			\sum_{x=0}^\infty \frac{1}{\lambda(x)} = \frac{1}{2} + \sum_{x=1}^\infty \frac{1}{3 \cdot 2^x}.
			\]
			
			Calcoliamo la serie \(\sum_{x=1}^\infty \frac{1}{3 \cdot 2^x}\):
			\[
			\sum_{x=1}^\infty \frac{1}{3 \cdot 2^x} = \frac{1}{3} \sum_{x=1}^\infty \frac{1}{2^x}.
			\]
			Notiamo che la serie è una geometrica e quindi ne conosciamo la convergenza\\
			\[
			\sum_{x=1}^\infty \frac{1}{2^x} = \frac{\frac{1}{2}}{1-\frac{1}{2}}=1
			\]
			Abbiamo in conclusione che la serie diventa\\
			\[
			\sum_{x=0}^\infty \frac{1}{\lambda(x)} = \frac{1}{2} + \sum_{x=1}^\infty \frac{1}{3 \cdot 2^x}=\frac{1}{2}+\frac{1}{3}=\frac{5}{6}
			\]
			Quindi la catena è esplosiva.
		\end{enumerate}
		\item \textbf{Esercizio 2: }
		\begin{enumerate}[label=\alph*)]
			\item Modelliamo come una CTMC, consideriamo un sistema con 6 stati ($n=0,1,2,3,4,5$) che rappresentano il numero di cartucce disponibili. Il tasso di transizione dipende dal numero di stampanti operative e dal numero di cartucce in ricarica:\\
			\begin{itemize}
				\item $q(n,n+1)=min(5-n,2) \cdot 1$\\ avviene quando una cartuccia viene ricaricata. 
				\item $q(n,n-1)=min(n,3) \cdot \frac{1}{6}$\\ avviene quando una stampante consuma una cartuccia, con tasso proporzionale al numero di stampanti operative.
			\end{itemize}
			Successivamente procedo a calcolare la matrice di transizione $Q$\\
			\[
			\mathbf{Q} =
			\begin{pmatrix}
				-2 & 2 & 0 & 0 & 0 & 0 \\
				\frac{1}{6} & \frac{-13}{6} & 2 & 0 & 0 & 0 \\
				0 & \frac{1}{3} & \frac{-7}{3} & 2 & 0 & 0 \\
				0 & 0 & \frac{1}{2} & \frac{-5}{2} & 2 & 0 \\
				0 & 0 & 0 & \frac{1}{2} & \frac{-3}{2} & 1 \\
				0 & 0 & 0 & 0 & \frac{1}{2} & \frac{-1}{2}
				
			\end{pmatrix}
			\]
			Ora possiamo calcolare la distribuzione invariante $\pi$ tali che 
			\[
			\pi Q=0
			\]
			e otteniamo\\
			\[
			\pi=(\frac{1}{3829},\frac{12}{3829},\frac{72}{3829},\frac{288}{3829},\frac{1152}{3829},\frac{2304}{3829})
			\]
			Possiamo affermare che tutte e 3 le stampanti lavoreranno assieme se $n\geq 3$ quindi:\\
			\[
			\frac{288}{3829} +\frac{1152}{3829}+\frac{2304}{3829}=\frac{3744}{3829}
			\]
			\item 
		\end{enumerate}
		\item \textbf{Esercizio 3: }
		\item \textbf{Esercizio 4: }
		\item \textbf{Esercizio 5: }
	\end{itemize}
\end{document}