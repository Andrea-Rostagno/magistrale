\documentclass[a4paper,12pt]{article}
\usepackage{graphicx}
\usepackage{verbatim}
\usepackage{amsmath}
\usepackage{enumitem}
\usepackage{amsfonts}
\usepackage{amssymb}
\usepackage{listings}
\usepackage{xcolor}




\begin{document}
	\title{\textbf{Homework2 Rostagno}}
	\author{295706}
	\date{\today}
	\maketitle
	
	\begin{itemize}
		\item \textbf{Esercizio 1: }
		\begin{enumerate}[label=\alph*)]
			\item Per prima cosa dobbiamo calcolare i rate della embedded DTMC:\\
			\[
			r(x,x+1)=\frac{2^x}{3 \cdot 2^x}=\frac{1}{3}
			\]
			\[
			r(x,x-1)=\frac{2 \cdot2^x}{3 \cdot 2^x}=\frac{2}{3}
			\]
			Siccome il sistema ha una forte tendenza a tornare verso stati inferiori e $r(x,x+1) <\frac{1}{2}$, allora possiamo dire che sará ricorrente.\\
			Sappiamo quindi che la catena sará non esplosiva.\\
			Per quanto riguarda la ricorrenza abbiamo che:
			\[
			M=\sum_{j=0}^{\infty}\prod_{i=0}^{j-1}\frac{2^i}{2^{i+2}}=\sum_{j=0}^{\infty}(\frac{1}{4})^j=\frac{4}{3}
			\]
			Dato che M é finito, esiste una distribuzione stazionaria che implica che la catena sia non esplosiva positiva ricorrente.
			\item Per prima cosa dobbiamo calcolare i rate della embedded DTMC:\\
			\[
			r(x,x+1)=\frac{2^x}{2 \cdot 2^x}=\frac{1}{2}
			\]
			\[
			r(x,x-1)=\frac{2^x}{2 \cdot 2^x}=\frac{1}{2}
			\]
			L'embedded DTMC è una catena simmetrica con probabilità di andare verso l'alto e verso il basso uguali, é quindi noto si tratti di una catena null-ricorrente .\\
			Sappiamo quindi che la catena sará non esplosiva.\\
		
			\item Per prima cosa dobbiamo calcolare i rate della embedded DTMC:\\
			\[
			r(x,x+1)=\frac{2^{2x+1}}{3 \cdot 2^{2x}}=\frac{2}{3}
			\]
			\[
			r(x,x-1)=\frac{2^{2x}}{3 \cdot 2^{2x}}=\frac{1}{3}
			\]
			Siccome il sistema ha una forte tendenza ad andare verso stati superiori e $r(x,x+1) >\frac{1}{2}$, allora possiamo dire che sará transiente.\\
			Per quanto riguarda la distribuzione abbiamo che:
			\[
			M=\sum_{j=0}^{\infty}\prod_{i=0}^{j-1}\frac{2^{2i+1}}{2^{2i+2}}=\sum_{j=0}^{\infty}(\frac{1}{2})^j=2
			\]
			Dato che M é finito, possiamo appliccare il teorema secondo cui se la eDTMC é transiente ed esiste una distribuzione, allora la catena sará esplosiva.
		\end{enumerate}
		\newpage
		\item \textbf{Esercizio 2: }
		\begin{enumerate}[label=\alph*)]
			\item Modelliamo come una CTMC, consideriamo un sistema con 6 stati ($n=0,1,2,3,4,5$) che rappresentano il numero di cartucce disponibili. Il tasso di transizione dipende dal numero di stampanti operative e dal numero di cartucce in ricarica:\\
			\begin{itemize}
				\item $q(n,n+1)=min(5-n,2) \cdot 1$\\ avviene quando una cartuccia viene ricaricata. 
				\item $q(n,n-1)=min(n,3) \cdot \frac{1}{6}$\\ avviene quando una stampante consuma una cartuccia, con tasso proporzionale al numero di stampanti operative.
			\end{itemize}
			Successivamente procedo a calcolare la matrice di transizione $Q$\\
			\[
			\mathbf{Q} =
			\begin{pmatrix}
				-2 & 2 & 0 & 0 & 0 & 0 \\
				\frac{1}{6} & \frac{-13}{6} & 2 & 0 & 0 & 0 \\
				0 & \frac{1}{3} & \frac{-7}{3} & 2 & 0 & 0 \\
				0 & 0 & \frac{1}{2} & \frac{-5}{2} & 2 & 0 \\
				0 & 0 & 0 & \frac{1}{2} & \frac{-3}{2} & 1 \\
				0 & 0 & 0 & 0 & \frac{1}{2} & \frac{-1}{2}
				
			\end{pmatrix}
			\]
			Ora possiamo calcolare la distribuzione invariante $\pi$ tali che 
			\[
			\pi Q=0
			\]
			e otteniamo\\
			\[
			\pi=(\frac{1}{3829},\frac{12}{3829},\frac{72}{3829},\frac{288}{3829},\frac{1152}{3829},\frac{2304}{3829})
			\]
			Possiamo affermare che tutte e 3 le stampanti lavoreranno assieme se $n\geq 3$ quindi:\\
			\[
			\frac{288}{3829} +\frac{1152}{3829}+\frac{2304}{3829}=\frac{3744}{3829}
			\]
			\item Iniziamo a calcolare il numero medio di stampanti operative giornaliero\\
			\[\text{Media stamp. operative}=\sum_{n=0}^{5}\text{(stamp. operative nello stato n)}\cdot\pi(n)
			\]
			Esplicitiamo la formula\\
			\[
			 \pi(0) \cdot 0 + \pi(1) \cdot 1 + \pi(2) \cdot 2 + \pi(3) \cdot 3 + \pi(4) \cdot 3 + \pi(5) \cdot 3
			\]
			Sostituendo i vari valori otteniamo\\
			\[
			0+\frac{12}{3829}+\frac{144}{3829}+\frac{864}{3829}+\frac{3456}{3829}+\frac{6912}{3829}=\frac{11388}{3829}
			\]
			Ora sappiamo che vengono stampate 1000 pagine al giorno per ogni stampante funzionante e ci interessa sapere all'anno quante ne vengono stampate, quindi\\
			\[
			\frac{11388}{3829}\cdot1000\cdot365=\frac{4156620000}{3829}=1.085.562 \text{pagine}
			\]
		\end{enumerate}
		\newpage
		\item \textbf{Esercizio 3: }
		\begin{enumerate} [label=\alph*)]
			\item Iniziamo con il definire i valori di $\lambda_1$ e $\lambda_2$ in modo da definire la traffic equation\\
			\[
			\lambda_1 = \lambda 
			\]
			\[
			\lambda_2=\lambda + (1-p)\lambda_2=\frac{\lambda}{p}
			\]
			Siccome ci troviamo in code del tipo (M/M/1), la condizione per cui siano positivamente ricorrenti è $\lambda < \mu$, nel nostro caso\\
			\[
			\lambda < \mu_1
			\]
			\[
			\lambda < p\mu_2
			\]
			Supponiamo di essere sotto tali condizioni, la distribuzione stazionaria sarà $\pi(n_1,n_2)=\pi_1(n_1)\pi_2(n_2)$, nello specifico\\
			\[
			\pi_1(n_1) = \left( 1 - \frac{\lambda}{\mu_1} \right) \left( \frac{\lambda}{\mu_1} \right)^{n_1}
			\]
			\[
			\pi_1(n_2) = \left( 1 - \frac{\lambda}{p\mu_2} \right) \left( \frac{\lambda}{p\mu_2} \right)^{n_2}
			\]
			\item In due ore, il numero di oggetti che passano l'ispezione segue un processo di Poisson con tasso:\\
			\[\lambda_{pass}=2 \cdot \lambda \cdot p
			\]
			Dobbiamo calcolare la probabilità che meno di 3 oggetti passino:\\
			\[
			P(\text{meno di 3 passano})=P(X<3)
			\]
			Usiamo la formula della distribuzione di Poisson:
			\[
			P(X < 3) = P(X = 0) + P(X = 1) + P(X = 2)
			\]
			\[
			P(X < 3) = e^{-\lambda_{\text{pass}}} \left( 1 + \lambda_{\text{pass}} + \frac{\lambda_{\text{pass}}^2}{2} \right).
			\]
			
			\item Iniziamo a calcolare il tasso effettivo di arrivo al centro macchine, ovvero:\\
			\[\lambda_{eff}=\lambda +(1-p)\frac{\lambda}{p}=\frac{\lambda}{p}
			\]
			Adesso posso calcolare il numero medio di pezzi al centro macchine\\
			\[
			\mathbb{E}[N_{\text{macchine}}] = \frac{\lambda_{\text{eff}}}{\mu_1 - \lambda_{\text{eff}}}.
			\]
			dove\\
			\begin{itemize}
				\item $\lambda = 10$ tasso di arrivo al centro macchina
				\item $p$ probabilità di passare l'ispezione 
				\item $\mu_1 = 15$ rate al centro macchina
			\end{itemize}
			
		\end{enumerate}
		\newpage
		\item \textbf{Esercizio 4: }
		\begin{enumerate} [label=\alph*)]
			\item Nell'urna ci sono $n \geq 2$ palline, alcune bianche (W) e alcune nere (B). Definiamo il numero totale di palline $n=W+B$ ricordando che rimane costante. Il gioco modifica il numero di palline bianche o nere ma mantiene inalterata la somma.\\
			Definiamo una funzione di variabilità come il prodotto
			\[ V= W \cdot B\]
			Questa quantità rappresenta il prodotto tra il numero di palline bianche e nere.
			\begin{itemize}
				\item \text{Caso 1:} Se vengono estratte due palline dello stesso colore, $W$ e $B$ rimangono invariati, quindi anche $V$ non cambia.
				\item \text{Caso 2:} Se vengono estratte due palline di colori diversi:
				\begin{itemize}
					\item $W$ aumenta di $1$ e $B$ diminuisce di $1$ (o viceversa) con uguale probabilità.
					\item Questo modifica $V$ come segue:
					$$
					V_{\text{nuovo}} = (W + 1) \cdot (B - 1) = W \cdot B + (B - W - 1).
					$$
				\end{itemize}
				Il cambiamento in $V$ è decrescente in media perché $B - W - 1$ è negativo quando $W$ e $B$ non sono bilanciati.
			\end{itemize}
			
			Quindi, la funzione $V$ decresce stocasticamente ad ogni iterazione, tranne quando $W$ o $B$ sono già massimi (cioè $V = 0$).\\
			\newline
			Gli stati $W = 0$ (tutte le palline nere) e $B = 0$ (tutte le palline bianche) sono stati \textit{assorbenti}, poiché non possono più verificarsi cambiamenti nel sistema. In questi stati, $V = 0$.\\
			\newline
			La quantità $V$ è una supermartingales perché decresce stocasticamente ad ogni iterazione, come dimostrato sopra.\\
			Per il teorema di convergenza delle closed martingales, un processo stocastico limitato e decrescente converge quasi sicuramente a un valore limite.\\
			Essendo che abbiamo due possibili stati assorbenti, il sistema convergerà necessariamente a uno di essi. \\
			Quindi possiamo affermare che alla fine del processo avremo le palline tutte dello stesso colore.
			
			\item La probabilità che tutte le palline diventino bianche $P(W=n)$ è proporzionale alla quantità iniziale di palline bianche rispetto al totale:
			\[P(W=n)=\frac{W_0}{W_0+B_0}=\frac{10}{10 + 20}=\frac{1}{3}\]
			\item Possiamo ricalcolare 
			$P(W = n \mid W_{\sigma} = 12)$ come se $W_{\sigma} = 12$ fosse il nuovo stato iniziale, con $n = 30$.\\
			\[P(W=n \mid W_\sigma =12)=\frac{W_\sigma}{W_\sigma+B_\sigma}=\frac{12}{12 + 18}=\frac{2}{5}\]
			
			
		\end{enumerate}
		\newpage
		\item \textbf{Esercizio 5: }
		
		L'equazione data è:
		\[
		\mathbb{E}[f(X_{n+1}) - f(X_n) \mid X_n] = \mathbb{E}[g(X_{n+1}) - g(X_n) \mid X_n].
		\]
		\text{Applichiamo la propietà della linearità:}
		\[
		\mathbb{E}[f(X_{n+1}) \mid X_n] - \mathbb{E}[f(X_n) \mid X_n]  = \mathbb{E}[g(X_{n+1}) \mid X_n] - \mathbb{E}[g(X_n) \mid X_n] .
		\]
		Semplifichiamo:
		\[
		\mathbb{E}[f(X_{n+1}) \mid X_n] - f(X_n) = \mathbb{E}[g(X_{n+1}) \mid X_n] - g(X_n).
		\]
		\text{Spostando i termini, otteniamo:}
		\[
		\mathbb{E}[f(X_{n+1}) \mid X_n] - \mathbb{E}[g(X_{n+1}) \mid X_n] = f(X_n) - g(X_n).
		\]
		\text{Definiamo:}
		\[
		h(X_n) = f(X_n) - g(X_n).
		\]
		\text{Quindi, l'equazione diventa:}
		\[
		\mathbb{E}[h(X_{n+1}) \mid X_n] = h(X_n).
		\]
		Questa equazione implica che $\left(h(X_n)\right)_{n=0}^\infty$ è un martingales rispetto alla filtrazione generata da $(X_n)$ in quanto ci troviamo in un DTMC ricorrente. Questo implica che il valore medio condizionato di $h(X_n)$ rimane costante.\\
		Dato che $(X_n)$ è ricorrente, possiamo concludere che $h(X_n)$ deve essere costante su $S$. Pertanto:
		$$
		h(x) = f(x) - g(x) = C,
		$$
		dove $C$ è una costante indipendente da $x$.\\
		Poiché $f(x) \geq g(x)$ per ogni $x \in S$, abbiamo:
		$$
		C = f(x) - g(x) \geq 0.
		$$
		Abbiamo dimostrato che:
		\[
		f(x)=g(x)+C
		\]
	
		
	\end{itemize}
\end{document}